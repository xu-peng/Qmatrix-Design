\documentclass{edm_template}
%\documentclass[11pt,twocolumn]{article}
\usepackage{amsmath}
\usepackage{algorithmic}
\usepackage{adjustbox}
\usepackage{paralist}
\usepackage{array}
\usepackage{apacite}
\usepackage{multirow}
\usepackage{dcolumn}
\usepackage{array}
\usepackage{kbordermatrix}
\usepackage{balance}
\usepackage{float}
\usepackage{soul}
\DeclareMathOperator*{\argmax}{arg\,max}

\newcommand{\Mem}[1]{{\hl{[#1]}}}

\begin{document}
\begin{abstract}
In recent years, the identifiability problem of Q-matrix under DINA/DINO model has been proposed and researched. Strict statistical discussion and formulation have been given. One useful result is that under the DINA and DINO models, with Q-matrix, slip and guess being known, the population proportion parameter $p$ is identifiable if and only if Q is complete (meaning it contains all unit vectors)\Mem{boolean unit vector?}. In order to diagnose students, this condition should be used as a basic requirement for Q-matrix design\Mem{Not always feasible}. However, it does not suffice by just using a small Q-matrix which merely satisfies identifiability condition to offer a correct diagnosis for each individual student, particularly because the condition is about the parameters for the whole population. Fortunately, by increasing the number of questions being asked, i.e the number of q-vectors, we can improve the accuracy of estimation on student profiles. The ways to achieve a balance between conciseness and diagnosability of Q-matrix are discussed empirically in this paper.

\Mem{How I see it and feel free to take any piece of the text!}

Title: Q-matrix Design for the DINA model and the identifiability property.

In most contexts of student skills assessment, whether the test material is administered by the teacher or within a learning environment, there is a strong incentive to minimize the number of questions or exercises administered in order to get an accurate assessment.  This minimization objective can be framed as a Q-matrix design problem: given a set of skills to assess and a fixed number of question items, determine the optimal set of items, out of a potentially large pool, that will yield the most accurate assessment.  In recent years, the Q-matrix identifiability under DINA/DINO models has been proposed as a guiding principle for that purpose.  We empirically investigate the extent to which identifiability can serve that purpose. Identifiability of Q-matrices is studied throughout a range of conditions in an effort to measure and understand its relation to student skills assessment.  We compare identifiability to other Q-matrix design principles through simulation studies of skills assessment with both synthetic and real data.

\end{abstract}
\end{document}
